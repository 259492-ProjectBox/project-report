\chapter{\ifproject%
\ifenglish Experimentation and Results\else การทดลองและผลลัพธ์\fi
\else%
\ifenglish System Evaluation\else การประเมินระบบ\fi
\fi
}

ในบทนี้จะทำการทดสอบการทำงานของระบบในฟังก์ชันหลักต่างๆ

\section{\ifenglish Functionality Testing\else การทดสอบฟังก์ชันการทำงาน\fi}

\subsection{\ifenglish User Login\else การเข้าสู่ระบบของผู้ใช้\fi}
\begin{itemize}
  \item \textbf{วัตถุประสงค์:} เพื่อตรวจสอบว่าผู้ใช้งานสามารถเข้าสู่ระบบได้โดยใช้บัญชี CMU Account
  \item \textbf{ขั้นตอนการทดสอบ:}
  \begin{enumerate}
    \item ไปยังหน้าเข้าสู่ระบบ
    \item ป้อนข้อมูลบัญชี CMU Account ที่ถูกต้อง
    \item คลิกปุ่ม "เข้าสู่ระบบ"
  \end{enumerate}
  \item \textbf{ผลลัพธ์ที่คาดหวัง:} ผู้ใช้งานสามารถเข้าสู่ระบบได้สำเร็จ และถูกเปลี่ยนเส้นทางไปยังหน้าแดชบอร์ด
  \item \textbf{ผลลัพธ์ที่ได้:} ผู้ใช้งานสามารถเข้าสู่ระบบได้สำเร็จ และถูกเปลี่ยนเส้นทางไปยังหน้าแดชบอร์ด
  \item \textbf{สถานะ:} ผ่าน
\end{itemize}

\subsection{\ifenglish Project Search\else การค้นหาโครงงาน\fi}

\subsubsection{\ifenglish Quick Search\else การค้นหาแบบ Quick Search\fi}
\begin{itemize}
  \item \textbf{วัตถุประสงค์:} เพื่อตรวจสอบว่าผู้ใช้งานสามารถค้นหาโครงงานได้โดยใช้ฟังก์ชันการค้นหาแบบ Quick Search
  \item \textbf{ขั้นตอนการทดสอบ:}
  \begin{enumerate}
    \item ไปยังหน้าค้นหา
    \item ป้อนคำค้นหาในช่องค้นหา (เช่น ชื่อโครงงาน, คำอธิบาย, รหัสนักศึกษา/ชื่อ, ชื่ออาจารย์ที่ปรึกษา ภาษาไทย/อังกฤษ)
    \item คลิกปุ่ม "ค้นหา"
  \end{enumerate}
  \item \textbf{ผลลัพธ์ที่คาดหวัง:} โครงงานที่เกี่ยวข้องกับคำค้นหาจะแสดงขึ้นมา
  \item \textbf{ผลลัพธ์ที่ได้:} โครงงานที่เกี่ยวข้องแสดงผลได้อย่างถูกต้อง แต่มีปัญหาคือไม่สามารถเว้นคำในการค้นหา เช่น ค้นหาคำว่า "Digital Platform" แต่ถ้าพิมพ์ "Digital" หรือ "Platform" จะหาเจอ
  \item \textbf{สถานะ:} ผ่านบางส่วน
\end{itemize}

\subsubsection{\ifenglish Advanced Search\else การค้นหาแบบ Advanced Search\fi}
\begin{itemize}
  \item \textbf{วัตถุประสงค์:} เพื่อตรวจสอบว่าผู้ใช้งานสามารถค้นหาโครงงานได้โดยใช้ฟังก์ชันการค้นหาแบบ Advanced Search
  \item \textbf{ขั้นตอนการทดสอบ:}
  \begin{enumerate}
    \item ไปยังหน้าค้นหา
    \item เลือกหมวดหมู่ต่างๆ สำหรับการค้นหา (เช่น หมวดหมู่โครงงาน, ปีการศึกษา, ชื่ออาจารย์ที่ปรึกษา)
    \item คลิกปุ่ม "ค้นหา"
  \end{enumerate}
  \item \textbf{ผลลัพธ์ที่คาดหวัง:} โครงงานที่เกี่ยวข้องกับหมวดหมู่ที่เลือกจะแสดงขึ้นมา
  \item \textbf{ผลลัพธ์ที่ได้:} โครงงานที่เกี่ยวข้องแสดงผลได้อย่างถูกต้อง
  \item \textbf{สถานะ:} ผ่าน
\end{itemize}

\subsubsection{\ifenglish PDF Search\else การค้นหาแบบ PDF Search\fi}
\begin{itemize}
  \item \textbf{วัตถุประสงค์:} เพื่อตรวจสอบว่าผู้ใช้งานสามารถค้นหาโครงงานได้โดยใช้ฟังก์ชันการค้นหาแบบ PDF Search
  \item \textbf{ขั้นตอนการทดสอบ:}
  \begin{enumerate}
    \item ไปยังหน้าค้นหา
    \item ป้อนคำค้นหาในช่องค้นหา
    \item คลิกปุ่ม "ค้นหา"
  \end{enumerate}
  \item \textbf{ผลลัพธ์ที่คาดหวัง:} โครงงานที่มีคำค้นหาในรายงาน PDF จะแสดงขึ้นมาและมีการไฮไลต์คำค้นหา
  \item \textbf{ผลลัพธ์ที่ได้:} โครงงานที่เกี่ยวข้องแสดงผลได้อย่างถูกต้อง แต่การค้นหาคำภาษาไทยยังมีผิดพลาดบ้าง เช่น ค้นหาคำว่า "บริหาร" อาจจะมีคำว่า "บริการ" มาด้วย ซึ่งคาดว่าปัญหาน่าจะมาจาก Elasticsearch
  \item \textbf{สถานะ:} ผ่านบางส่วน
\end{itemize}

\subsubsection{\ifenglish Keyword Search\else การค้นหาแบบ Keyword Search\fi}
\begin{itemize}
  \item \textbf{วัตถุประสงค์:} เพื่อตรวจสอบว่าผู้ใช้งานสามารถค้นหาโครงงานได้โดยใช้ฟังก์ชันการค้นหาแบบ Keyword Search
  \item \textbf{ขั้นตอนการทดสอบ:}
  \begin{enumerate}
    \item ไปยังหน้าค้นหา
    \item ป้อนคำค้นหาในช่องค้นหา
    \item คลิกปุ่ม "ค้นหา"
  \end{enumerate}
  \item \textbf{ผลลัพธ์ที่คาดหวัง:} โครงงานที่เกี่ยวข้องกับคำค้นหาจะแสดงขึ้นมา
  \item \textbf{ผลลัพธ์ที่ได้:} โครงงานที่เกี่ยวข้องแสดงผลได้อย่างถูกต้อง
  \item \textbf{สถานะ:} ผ่าน
\end{itemize}
\subsection{\ifenglish Project Creation\else การสร้างโครงงาน\fi}
\begin{itemize}
  \item \textbf{วัตถุประสงค์:} เพื่อตรวจสอบว่าผู้ใช้งานสามารถสร้างโครงงานใหม่ได้
  \item \textbf{ขั้นตอนการทดสอบ:}
  \begin{enumerate}
    \item ไปยังหน้าแดชบอร์ด
    \item คลิกปุ่ม "Create Project"
    \item กรอกรายละเอียดโครงงาน
    \item คลิกปุ่ม "Save"
  \end{enumerate}
  \item \textbf{ผลลัพธ์ที่คาดหวัง:} โครงงานถูกสร้างสำเร็จ และแสดงในหน้าแดชบอร์ด
  \item \textbf{ผลลัพธ์ที่ได้:} โครงงานถูกสร้างสำเร็จและแสดงผล
  \item \textbf{สถานะ:} ผ่าน
\end{itemize}
