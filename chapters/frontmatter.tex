\maketitle
\makesignature

\ifproject
\begin{abstractTH}
% เขียนบทคัดย่อของโครงงานที่นี่
% แพลตฟอร์มนี้ถูกพัฒนาขึ้นเพื่อรวบรวมและจัดเก็บข้อมูลโครงงานวิศวกรรมศาสตร์ของนักศึกษามหาวิทยาลัยเชียงใหม่ โดยมีวัตถุประสงค์หลักในการสร้างแหล่งข้อมูลกลางที่เป็นระบบและเข้าถึงได้ง่าย นักศึกษาสามารถค้นหาและศึกษาโครงงานของรุ่นพี่ เพื่อนำไปเป็นแนวทางหรือแรงบันดาลใจในการพัฒนาโครงงานของตนเอง ระบบถูกออกแบบให้มีการจัดหมวดหมู่และฟังก์ชันการค้นหาที่มีประสิทธิภาพ เพื่อให้การเข้าถึงข้อมูลเป็นไปอย่างสะดวกและรวดเร็ว แพลตฟอร์มนี้จึงเป็นเครื่องมือสำคัญในการสนับสนุนการเรียนรู้และการพัฒนานวัตกรรมของนักศึกษาวิศวกรรมศาสตร์

แพลตฟอร์มนี้ถูกพัฒนาขึ้นโดยมีจุดประสงค์หลักเพื่อเก็บรวบรวมโครงงานวิศวกรรม
ของนักศึกษาคณะวิศวกรรมศาสตร์มหาวิทยาลัยเชียงใหม่ และใช้ในการสร้างแหล่งข้อมูลกลางที่เป็นระบบและเข้าถึงได้ง่าย
\enskip นักศึกษาสามารถค้นหาและศึกษาโครงงานของรุ่นพี่ เพื่อนำไปใช้เป็นแรงบันดาลใจหรือตัวอย่างในการทำโครงงาน \enskip
โดยระบบได้ออกแบบให้สามารถค้นหาได้จากหมวดหมู่ต่างๆ หรือสามารถค้นหาโครงงานที่เกี่ยวห้องได้โดยต้นหาจาก keyword ที่เกี่ยวข้องใน pdf ของโครงงานนั้นๆ

% การเขียนรายงานเป็นส่วนหนึ่งของการทำโครงงานวิศวกรรมคอมพิวเตอร์
% เพื่อทบทวนทฤษฎีที่เกี่ยวข้อง อธิบายขั้นตอนวิธีแก้ปัญหาเชิงวิศวกรรม และวิเคราะห์และสรุปผลการทดลองอุปกรณ์และระบบต่างๆ
% \enskip อย่างไรก็ดี การสร้างรูปเล่มรายงานให้ถูกรูปแบบนั้นเป็นขั้นตอนที่ยุ่งยาก
% แม้ว่าจะมีต้นแบบสำหรับใช้ในโปรแกรม Microsoft Word แล้วก็ตาม
% แต่นักศึกษาส่วนใหญ่ยังคงค้นพบว่าการใช้งานมีความซับซ้อน และเกิดความผิดพลาดในการจัดรูปแบบ กำหนดเลขหัวข้อ และสร้างสารบัญอยู่
% \enskip ภาควิชาวิศวกรรมคอมพิวเตอร์จึงได้จัดทำต้นแบบรูปเล่มรายงานโดยใช้ระบบจัดเตรียมเอกสาร
% \LaTeX{} เพื่อช่วยให้นักศึกษาเขียนรายงานได้อย่างสะดวกและรวดเร็วมากยิ่งขึ้น
\end{abstractTH}

\begin{abstract}
\hspace{1.27cm}This platform was developed with the main purpose of collecting engineering projects of students of the Faculty of Engineering, Chiang Mai University, and used to create a centralized and easily accessible source of information.
\enskip Students can search for and study the projects of their seniors to use as inspiration or examples for doing projects. \enskip
The system is designed to be searchable by various categories or to search for projects related to the room by searching for relevant keywords in the PDF of that project.
\end{abstract}

\iffalse
\begin{dedication}
This document is dedicated to all Chiang Mai University students.

Dedication page is optional.
\end{dedication}
\fi % \iffalse

\begin{acknowledgments}
% Your acknowledgments go here. Make sure it sits inside the
 \hspace{1.27cm}โครงงานนี้จะไม่สามารถสำเร็จได้ถ้าไม่ได้ความกรุณาจาก ผศ. โดม โพธิกานนท์ อาจารย์ที่ปรึกษาโครงงาน ที่ได้สละเวลาให้ความช่วยเหลือให้คำแนะนำและสนับสนุนในการทำโครงงานนี้รวมถึงอ.ดร.ชินวัตร อิศราดิสัยกุล  และ ผศ.ดร.นิรันดร์ พิสุทธอานนท์ ที่ ให้คำปรึกษาจนทำให้โครงงานเล่มนี้เสร็จ
สมบูรณ์ไปได้
% \texttt{acknowledgment} environment.


\acksign{2024}{2}{2}
\end{acknowledgments}%
\fi % \ifproject

\contentspage

\ifproject
\figurelistpage

\tablelistpage
\fi % \ifproject

% \abbrlist % this page is optional

% \symlist % this page is optional

% \preface % this section is optional
